%------------------Settings-------------------------

\documentclass[12pt]{article}
\usepackage[utf8]{inputenc}
\usepackage[russian]{babel}
\usepackage{amsmath,amssymb}
\usepackage{graphics}
\usepackage{pbox}
\usepackage[x11names]{xcolor}
\definecolor{brightmaroon}{rgb}{0.76, 0.13, 0.28}
\definecolor{royalazure}{rgb}{0.0, 0.22, 0.66}
\usepackage[colorlinks=true,linkcolor=royalazure]{hyperref}
\usepackage{tikz, tkz-fct, pgfplots}
\usetikzlibrary{arrows}
\usepackage{geometry}
\geometry{
	a4paper,
	total={170mm,257mm},
	left=20mm,
	top=20mm
} 
\usepackage[labelsep=period]{caption}

% ----------------- Commands ----------------- 

\newcommand{\eps}{\varepsilon}
\newcommand\tline[2]{$\underset{\text{#1}}{\text{\underline{\hspace{#2}}}}$}

% ----------------- Set graphics path ----------------- 
\graphicspath{{img/}}
\begin{document}
	\pagestyle{empty}
	
	% ----------------------Title----------------------------------
	\centerline{\large Министерство науки и высшего образования}	
	\centerline{\large Федеральное государственное бюджетное образовательное}
	\centerline{\large учреждение высшего образования}
	\centerline{\large ``Московский государственный технический университет}
	\centerline{\large имени Н.Э. Баумана}
	\centerline{\large (национальный исследовательский университет)''}
	\centerline{\large (МГТУ им. Н.Э. Баумана)}
	\hrule
	\vspace{0.5cm}
	\begin{figure}[h]
		\center
		\includegraphics[height=0.35\linewidth]{bmstu-logo-color.jpg}
	\end{figure}
	\begin{center}
		\large	
		\begin{tabular}{c}
			Факультет ``Фундаментальные науки'' \\
			Кафедра ``Высшая математика''		
		\end{tabular}
	\end{center}
	\vspace{0.5cm}
	\begin{center}
		\LARGE \bf	
		\begin{tabular}{c}
			\textsc{Отчёт} \\
			по учебной практике \\
			за 1 семестр 2020---2021 гг.
		\end{tabular}
	\end{center}
	\vspace{0.5cm}
	\begin{center}
		\large
		\begin{tabular}{p{5.3cm}ll}
			\pbox{5.45cm}{
				Руководитель практики,\\
				ст. преп. кафедры ФН1} 	& \tline{\it(подпись)}{5cm} & Кравченко О.В. \\[0.5cm]
			студент группы ФН1--11 		& \tline{\it(подпись)}{5cm} & Скалабо И.В.
		\end{tabular}
	\end{center}
	\vfill
	\begin{center}
		\large	
		\begin{tabular}{c}
			Москва, \\
			2020 г.
		\end{tabular}
	\end{center}
	\newpage
	\newpage	
	\tableofcontents
	%------------------Table of contents----------------------
	\newpage
	\section{Цели и задачи практики}	
	\subsection{Цели}
	--- развитие компетенций, способствующих успешному освоению материала бакалавриата и необходимых в будущей профессиональной деятельности.
	\subsection{Задачи}
	\begin{enumerate}
		\item Знакомство с программными средствами, необходимыми в будущей профессиональной деятельности.
		\item Развитие умения поиска необходимой информации в специальной литературе и других источниках.
		\item Развитие навыков составления отчётов и презентации результатов.
	\end{enumerate}
	\subsection{Индивидуальное задание}	
	\begin{enumerate}
		\item Изучить способы отображения математической информации в системе вёртски \LaTeX.
		\item Изучить возможности  системы контроля версий \textsf{Git}.
		\item Научиться верстать математические тексты, содержащие формулы и графики в системе \LaTeX.
		Для этого, выполнить установку свободно распространяемого дистрибутива \textsf{TeXLive} и оболочки \textsf{TeXStudio}.
		\item Оформить в системе \LaTeX типовые расчёты по курсу математического анализа согласно своему варианту.
		\item Создать аккаунт на онлайн ресурсе \textsf{GitHub} и загрузить исходные \textsf{tex}--файлы 
		и результат компиляции в формате \textsf{pdf}.
	\end{enumerate} 
	%---------------------------------------------------------------
	\newpage
	\section{Отчёт}
	Актуальность темы продиктована необходимостью владеть системой вёрстки \LaTeX и средой вёрстки \textsf{TeXStudio} для
	отображения текста, формул и графиков. Полученные в ходе практики навыки могут быть применены при написании
	курсовых проектов и дипломной работы, а также в дальнейшей профессиональной деятельности.
	Ситема вёрстки \LaTeX содержит большое количество инструментов (пакетов), упрощающих отображение информации в различных 
	сферах инженерной и научной деятельности. 
	%-----------------------------------------------------------------
	\newpage
	\section{Индивидуальное задание}
	%\subsection{Элементарные функции и их графики.}
	%\input{src/part1.tex}
	%==============================================================================
	\subsection{Пределы и непрерывность.}
	%---------------------------- Problem 1----------------------------------
	\subsubsection*{\center Задача № 1.}
	{\bf Условие.~}
	Дана последовательность $a_{n}=\dfrac{4n-3}{2n+1}$ и число $c={2} $. Доказать, что $\lim\limits_{x\rightarrow\infty} a_{n}=c $, а именно, для каждого $\varepsilon>0$ найти наименьшее натуральное число  $N{=}N(\varepsilon)$ такое, что $|a_{n}-c|<\varepsilon$ для всех $n>N(\varepsilon)$. Заполнить таблицу: 
	\begin{center}
		\begin{tabular}{ | p{25pt} | c | c | c | c |}
			\hline
			$\varepsilon$& $0{,}1$ & $0{,}01$ & $0{,}001$ \\ \hline
			$N(\varepsilon)$ &   &   &\\
			\hline
		\end{tabular}
	\end{center}
	\medskip
	%=====================================================================
	{\bf Решение.~}
	Рассмотрим неравенство $a_{n}-c<\varepsilon$, $\forall\varepsilon>0$, учитывая выражение для $a_{n}$ и $c$ из условия варианта, получим 
	$$\left|\frac{4n-3}{2n+1}-{2}\right|<\varepsilon$$
	Неравенство запишем в виде двойного неравенства и приведём выражение под знаком модуля к общему знаменателю, получим
	$${-}\varepsilon <{-}\dfrac{5}{2n+1}<\varepsilon$$
	Заметим, что правое неравенство выполнено для любого номера $n\in \mathbb{N}$ поэтому, будем рассматривать левое неравенство
	$$\frac{5}{2n+1}<\varepsilon$$
	Выполнив цепочку преобразований, перепишем неравенство относительно $n$, и, учитывая, что $n\in \mathbb{N}$, получим 
	$$\dfrac{5}{2n+1}<\varepsilon,$$
	$$2n+1>\dfrac{5}{\varepsilon},$$
	$$2n>\dfrac{5}{\varepsilon}-{1},$$
	$$n>\dfrac{5}{2\varepsilon}-\dfrac{1}{2},$$
	$$N(\varepsilon)=\biggl[\dfrac{5}{2\varepsilon}-\dfrac{1}{2}\biggr],$$
	где $[\;]$ -- целая часть от числа. Заполним таблицу:
	\begin{center}
		\begin{tabular}{ | p{25pt} | c | c | c | c |}
			\hline
			$\varepsilon$& $0{,}1$ & $0{,}01$ & $0{,}001$ \\ \hline
			$N(\varepsilon)$ & 24  & 249 & 2499\\
			\hline
		\end{tabular}
	\end{center}
	{\bf Проверка:~}
	$$|a_{25}-c|=\dfrac{5}{51}<0{,}1,$$
	$$|a_{250}-c|=\dfrac{5}{501}<0{,}01,$$
	$$|a_{2500}-c|=\dfrac{5}{5001}<0{,}001.$$
	\newpage
	% ---------------------------- Problem 2----------------------------------
	\subsubsection*{\center Задача № 2.}
	{\bf Условие.~}
	Вычислить пределы функций
	$$
	\begin{array}{cc}
		\text{\bf(а):} &  \lim\limits_{x\rightarrow-1}\dfrac{x^3-3x-2}{(x+x^2)(x+1)} , \\[10pt]
		\text{\bf(б):} & \lim\limits_{x\rightarrow+\infty} \dfrac{3x^2-\sqrt{x^3+x^6}}{\sqrt[3]{x^3+1}-\sqrt{2x^5-7x^3}} ,\\[10pt]
		\text{\bf(в):} & \lim\limits_{x\rightarrow+16} \dfrac{\sqrt[3]{\dfrac{x}{2}}-2}{\sqrt{x}-4},\\[10pt]
		\text{\bf(г):} & \lim\limits_{x\rightarrow0} (\cos{x})^{\dfrac{\ctg{x}}{\sin{2x}}}, \\[10pt]
		\text{\bf(д):} & \lim\limits_{x\rightarrow0-} \left(\dfrac{3+x}{\log_2 {(4-x)}}\right)^{\arcctg{\frac{1}{x}}} , \\[10pt]
		\text{\bf(е):}  & \lim\limits_{x\rightarrow+1} \dfrac{\sin{7 \pi x}}{\sin{8 \pi x}} . \\
	\end{array}
	$$
	\\
	{\bf Решение.~}\\
	\\
	\text{\bf(а):}
	$$
	\begin{array}{l}
		\lim\limits_{x\rightarrow-1}\dfrac{x^3-3x-2}{(x+x^2)(x+1)} =\left[\dfrac{0}{0} \right]= \lim\limits_{x\rightarrow-1}  \dfrac{(x+1)^2(x-2)}{(x+1)^2 x} = \lim\limits_{x\rightarrow-1}  \dfrac{x-2}{x}=\\ \medskip{}{}\\ \medskip{}{}=\dfrac{(-1)-2}{-1}={3}
	\end{array}
	$$
	\\
	\text{\bf(б):}
	$$
	\begin{array}{l}
		\lim\limits_{x\rightarrow+\infty} \dfrac{3x^2-\sqrt{x^3+x^6}}{\sqrt[3]{x^3+1}-\sqrt{2x^5-7x^3}}=\lim\limits_{x\rightarrow+\infty} \dfrac{3-\sqrt{\dfrac{1}{x}+x^2}}{\sqrt[3]{\dfrac{1}{x^3}+\dfrac{1}{x^6}}-\sqrt{2x-\dfrac{7}{x}}}=+\infty
		\\ \medskip{}{}
	\end{array}
	$$
	\text{\bf(в):}
	$$
	\begin{array}{l} 
		\lim\limits_{x\rightarrow+16} \dfrac{\sqrt[3]{\dfrac{x}{2}}-2}{\sqrt{x}-4} = \left[\dfrac{0}{0} \right]= \lim\limits_{x\rightarrow+16} \dfrac{\bigg(\sqrt[3]{\dfrac{x}{2}}-2 \bigg) \bigg(4+2^{\frac{2}{3}} \sqrt[3]{x}+\sqrt[3]{\dfrac{x^2}{4}}\bigg)}{\bigg(\sqrt{x}-4 \bigg) \bigg (4+2^{\frac{2}{3}} \sqrt[3]{x}+\sqrt[3]{\dfrac{x^2}{4}}\bigg )}=\\ \medskip{}{} \\ \medskip{}{} = \lim\limits_{x\rightarrow+16} \dfrac{\sqrt{x}+4}{2 \bigg (4+2^{\frac{2}{3}} \sqrt[3]{x}+\sqrt[3]{\dfrac{x^2}{4}} \bigg )}=\dfrac{8}{24}=\dfrac{1}{3}
	\end{array}
	$$
	\\
	\text{\bf(г):}
	$$
	\begin{array}{l}
		\lim\limits_{x\rightarrow0} (\cos{x})^{\frac{\ctg{x}}{\sin{2x}}}=\lim\limits_{x\rightarrow0} (\cos{x})^{\frac{1}{2\sin^2{x}}}=\left[1^{\infty}\right]=\lim\limits_{x\rightarrow0} (\sqrt{1-\sin^2{x}})^{\frac{1}{2\sin^2{x}}}=\\ \medskip{}{} \\ \medskip{}{}= \lim\limits_{x\rightarrow0} ({1-\sin^2{x}})^{\frac{1}{4\sin^2{x}}}=e^{-\frac{1}{4}}
	\end{array}
	$$
	\\
	\text{\bf(д):}
	$$
	\begin{array}{l}
		\lim\limits_{x\rightarrow0-} \left(\dfrac{3+x}{\log_2 {(4-x)}}\right)^{\arcctg{\frac{1}{x}}}= \lim\limits_{x\rightarrow0-} \left(\dfrac{3}{\log_2 {4}}\right)^{\arcctg{\frac{1}{0}}}= {(\dfrac{3}{2})}^{0}=1
		\\ \medskip{}{}
	\end{array}
	$$
	\text{\bf(е):}
	$$
	\begin{array}{l}
		\lim\limits_{x\rightarrow+1} \dfrac{\sin{7 \pi x}}{\sin{8 \pi x}}=\left[\dfrac{0}{0} \right]=\lim\limits_{x\rightarrow+1} \dfrac{7 \pi \cos{(7 \pi x)}}{8 \pi \cos{(8 \pi x)}}=-\dfrac{7}{8}
	\end{array}
	$$
	% ---------------------------- Problem 3----------------------------------
	\subsubsection*{\center Задача № 3.}
	{\bf Условие.~}\\
	\text{\bf(а):} Показать, что данные функции
	$f(x)$ и $g(x)$ являются бесконечно малыми или бесконечно большими
	при указанном стремлении аргумента. \\
	\text{\bf(б):} Для каждой функции $f(x)$ и $g(x)$ записать главную часть
	(эквивалентную ей функцию)  вида $C(x-x_0)^{\alpha}$ при $x\rightarrow x_0$ или $Cx^{\alpha}$
	при $x\rightarrow\infty$, указать их порядки малости (роста). \\
	\text{\bf(в):} Сравнить функции $f(x)$ и $g(x)$ при указанном стремлении.
	\begin{center}
		\begin{tabular}{|c|c|c|}
			\hline
			№ варианта & функции $f(x)$ и $g(x)$ & стремление \\[6pt]
			\hline
			22 & $f(x) = \dfrac{\ln{x}}{(1-x)^2},~g(x)=\dfrac{1}{1-\cos\sqrt{x-1}}$ & $x\rightarrow1+$ \\
			\hline
		\end{tabular}
		\bigskip
		\\
		{\bf Решение.~}\\
	\end{center}
	\medskip
	\text{\bf(а):}~Покажем, что $f(x)$ и $g(x)$ бесконечно большие функции,
	$$
	\begin{array}{l} 
		\lim\limits_{x\rightarrow 1+} f(x)=\lim\limits_{x\rightarrow 1+} \dfrac{\ln{x}}{(1-x)^2}=\left|\ln {x} \sim (x-1) \right|=\lim\limits_{x\rightarrow 1+} \dfrac{(x-1)}{(1-x)^2}=\lim\limits_{x\rightarrow 1+} \dfrac{1}{(x-1)}=\left[\dfrac{1}{+0} \right]=+\infty , 
		\\ \medskip{}{}
		\\ \medskip{}{}
		\lim\limits_{x\rightarrow 1+} g(x)= \lim\limits_{x\rightarrow 1+}  \dfrac{1}{1-\cos\sqrt{x-1}}=\left[\dfrac{1}{1-(1-0)} \right]=\left[\dfrac{1}{+0} \right]=+\infty.
	\end{array}
	$$
	\text{\bf(б):}~Так как $f(x)$ и $g(x)$ бесконечно большие функции, то эквивалентными им будут функции вида 
	$C(x-x_0)^{\alpha}$ при $x\rightarrow 1+$. Найдём эквивалентную для $f(x)$ из условия
	$$
	\lim\limits_{x\rightarrow 1+}\dfrac{f(x)}{(x-x_0)^{\alpha}} = C,
	$$
	где $C$ --- некоторая константа. Рассмотрим предел
	$$
	\lim\limits_{x\rightarrow 1+} \dfrac{\ln{x}}{(x-1)^{\alpha} (1-x)^2}=\lim\limits_{x\rightarrow 1+} \dfrac{(x-1)}{(x-1)^{\alpha}(1-x)^2}=\lim\limits_{x\rightarrow 1+} \dfrac{1}{(x-1)^{\alpha+1}}
	$$
	При $\alpha=-1 $ последний предел равен $1$, отсюда $C=1$ и 
	$$
	f(x)\sim \dfrac{1}{x-1} ~\text{при}~x\rightarrow 1+.
	$$
	Аналогично, рассмотрим предел
	$$
	\begin{array}{l}
		\lim\limits_{x\rightarrow 1+}\dfrac{g(x)}{(x-x_0)^{\alpha}}=\lim\limits_{x\rightarrow 1+} \dfrac{1}{(x-1)^{\alpha}(1-\cos\sqrt{x-1})} =\lim\limits_{x\rightarrow 1+} \dfrac{1}{(x-1)^{\alpha}(2\sin^2\dfrac{\sqrt{x-1}}{2})}=\\ \medskip{}{} \\ \medskip{}{}=\left|2\sin^2\dfrac{\sqrt{x-1}}{2} \sim \dfrac{x-1}{2} \right|=\lim\limits_{x\rightarrow 1+} \dfrac{2}{(x-1)^{\alpha+1}}
	\end{array}
	$$
	При $\alpha=-1$ последний предел равен $2$, отсюда $C=2$ и
	$$
	g(x)\sim\dfrac{2}{x-1}~\text{при}~x\rightarrow 1+.
	$$
	\text{\bf(в):}~Для сравнения функций $f(x)$ и $g(x)$ рассмотрим предел их отношения при указанном стремлении
	$$
	\lim\limits_{x\rightarrow 1+}\dfrac{f(x)}{g(x)}.
	$$
	Применим эквивалентности, определенные в пункте (б), получим
	$$
	\lim\limits_{x\rightarrow 1+}\dfrac{f(x)}{g(x)} = \lim\limits_{x\rightarrow 1+}\dfrac{\dfrac{1}{x-1}}{\dfrac{2}{x-1}}=\dfrac{1}{2}
	$$
	Отсюда, $f(x)$ и $g(x)$ есть бесконечно большие функции одного и того же порядка малости.
	%=================================================================================================================================
	%\subsection{Приложения дифференциального исчисления.}
	%\input{src/part3.tex}
	\newpage
	\addcontentsline{toc}{section}{Список литературы}
	\begin{thebibliography}{99}
		\bibitem{book01} Львовский С.М. Набор и вёрстка в системе \LaTeX, 2003 c.
		\bibitem{book02} Котельников И.А., Чеботаев П.3. \LaTeX по-русски.
	\end{thebibliography}
\end{document}